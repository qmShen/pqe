\chapter{Conclusion and future work}
\label{chap:conclusion}

\section{Conclusion}
In this survey, we first reviewed the diffusion of innnovation theory. Then we introduced the three classical problems: influence maximization, community detection, and link prediction. According to the gerneral order of data mining procedures, we introduced the preprocessing techniques for the colleceted social media data, the models describing diffusing process, the proposed algorithms, and the mining results of those three problems. In the last, we introduced serveral studies on visualizing the social network and diffusion process. We also surveyed some highly related visualization techniques which could be used for future work. 

The diffusion of innovation on social media currently is still a very hot topic in different areas: The econimists are trying to study the strategies of advertising; The sociologists are trying to study the individual behaviours as well as social behaviours for the social network. Data mining techniques are very powerful for processing the large scale of data and finding what we want like the most influential nodes. However, the data mining techniques cannot meet all our needs: a) Are there some unsual patterns when the diffusion happens? b) How should we inspect the diffsuion process on the network? Moreover, a common problem of the data mining techniques is how we could help users including the experts in other fields to understand the mining results. 

\section{Future work}

As we illustrated before, the visualization techniques have shown great power for users to quickly understand massive data and inspect patterns. However, so far, there are few visualization tools which can show the diffusion process on the social network. Thus, designing a new visualization for the information diffusion on social media is critical. Besides, in order to let users easily understand the data as well as  the mining results, just a new visualization is not enough: a visual analysis system is needed.

For this new visualization, first, we should consider, which information should be encoded as the node position for the graph. As in many cases in social networks, we usually don't have or need geographic information, but people are very sensitive to the position of a node. In addition, for visualizing the information diffusion on the social network, in order to make it easy for users to understand the trend of the diffusion, the positions of the nodes are important. Second, we need to consider in which form we should encode the temporal information. Although there are some visualizations that try to visualize the temporal information, they may not be suitable for our cases and the effects are not good enough. Third, as the data acquired from the social media are in a huge scale, the scalability of the visualization should be seriously concerned. For example, we could design multiple visualizations to visualize the data in different levels of details. 


For the visual analysis system, we should focus on two things: 1) how to visualize the mining results; 2) how to design the user interactions. As making users better understand the mining result, such as the top influencers and the communities, we should further encode the information that why the data mining techniques return such the results, which would greatly help the users to know the rationality of the results. Besides, we should consider the overlapping problems when we want to visualize the multiple mining results. The user interactions are designed to make it possible for the user to manipulate the data including the raw data, the preprocessed data and the mining results. For the raw data, we should allow users filtering and reloading operations; For the preprocessed data, considering the accuracy problems of the text mining, we should allow users refining the data, such as the structure of the network, the topic of the text and the sentiment of text (negative or positive). For the mining results, we'd better provide two ways for users: 1) modifying the parameters of the mining models then running the mining algorithm again; 2) modifying the mining result directly. 








