\abstract
\label{chap:abs}

Urban mobility indicates the human movement patterns in a city. With the rapid urbanization, mobility gradually raises more and more attention on a variety of researches areas. The studying of mobility is beneficial to both individual residents and city management, and it has a long history. Previous researches on the mobility are greatly limited to data viable. Thanks to the sensing technologies, more and more types of data can be collected to describe the human movement, providing new chances to study the mobility from different perspectives.  

Currently, many automatic methods have been proposed on the mobility and a lot of meaningful results has been found. However, the mobility patterns differ from time, space and situations, which greatly needs the involvement of the domain experts. Visual analytics bridge the gap between the techniques and domain knowledge.
  
In this survey, we first summarize the data modeling methods for visual analytics from a new perspective as well as the corresponding research challenges. Then we will focus on the general techniques used in different stages of analysis. After that, we classify the previous work based on common application problem and discuss how the problem is solved through the specific modeling and techniques.  At the end, we conclude with some future direction of mobility research.


\endabstract 
